\newcommand{\pdftitel}{PDFTITEL}
\newcommand{\autor}{Autor}
\newcommand{\arbeit}{ART DER ARBEIT}
\input{ads/header}

% Ab jetzt können auch Umlaute verwendet werden

\newcommand{\titel}{TITEL}
\newcommand{\martrikelnr}{Matrikelnr}
\newcommand{\kurs}{Kurs}
\newcommand{\datumAbgabe}{01.09.2014}
\newcommand{\firma}{Firma}
\newcommand{\firmenort}{Firmenort}
\newcommand{\abgabeort}{Stuttgart}
\newcommand{\abschluss}{Bachelor of Science}
\newcommand{\studiengang}{Wirtschaftsinformatik}
\newcommand{\dhbw}{Stuttgart}
\newcommand{\betreuertitel}{Betreuertitel}
\newcommand{\betreuerfirma}{Betreuername}
\newcommand{\funktion}{Betreuerfunktion}
\newcommand{\betreuer}{DH-Betreuer}
\newcommand{\zeitraum}{12 Wochen}
\newcommand{\arbeitsart}{1.Projektarbeit}
\newcommand{\dnd}{D\& D}

\begin{document}

	% Deckblatt
	\begin{spacing}{1}
	\setlength{\hoffset}{-10mm}
		\begin{titlepage}
	\begin{longtable}{p{.4\textwidth} p{.4\textwidth}}
	\end{longtable}
	%\enlargethispage{20mm}
	\begin{center}
	\begin{doublespace}
	  \vspace*{12mm}	{\LARGE\bf \titel }\\
	  \vspace*{12mm}	{\large\bf \arbeit}\\
	\end{doublespace}  
	  \vspace*{12mm}	vorgelegt am \datumAbgabe\\
	  \vspace*{12mm}	Fakultät Wirtschaft\\
	  \vspace*{3mm} 	Studiengang \studiengang\\
	 \vspace*{3mm} 	{\kurs}\\
	  \vspace*{12mm}	von\\
	  \vspace*{3mm} 	{\large\bf \autor}\\
	  
	  
	\end{center}
	\vfill
	\begin{spacing}{1.2}
	\begin{tabbing}
		mmmmmmmmmmmmmmmmmmmmmmmmmm     \= \kill
%		\textbf{Bearbeitungszeitraum}  \>  \zeitraum\\
		\textbf{Ausbildungsstätte:}	   \> \textbf{DHBW Stuttgart:} \\
		\firma						   \>  \\
		\betreuertitel				   \>  \\
		\betreuerfirma				   \>  \betreuer\\
		\funktion				   \>  \\
	\end{tabbing}
	\end{spacing}
	
	%Vertraulichkeitsvermerk bei Bedarf auskommentieren
	\begin{center}
	\textbf{Vertraulich}\\
	Der Inhalt der Arbeit darf Dritten ohne Genehmigung der Ausbildungsstätte nicht zugänglich gemacht werden.
	\end{center}
	
\end{titlepage}

	\end{spacing}
	\newpage

	\renewcommand{\thepage}{\Roman{page}}
	\setcounter{page}{1}

	\pagestyle{plain}

	% Inhaltsverzeichnis
	\begin{spacing}{1.1}
		\setcounter{tocdepth}{1}
		%für die Anzeige von Unterkapiteln im Inhaltsverzeichnis
		\setcounter{tocdepth}{2}
		\tableofcontents
	\end{spacing}
	\newpage
	
	% Abkürzungsverzeichnis
	\cleardoublepage
	\phantomsection \label{listofacs}
	\addcontentsline{toc}{chapter}{Abkürzungsverzeichnis}
	\chapter*{Abkürzungsverzeichnis}
%nur verwendete Akronyme werden letztlich im Dokument angezeigt
\begin{acronym}[YTMMM]
\setlength{\itemsep}{-\parsep}

\acro{Apps}{Applikationen}
\acro{CSS}{Cascasding Style Sheets}
\acro{HTML5}{Hypertext Markup Language 5}
\acro{XML}{Extensible Markup Language}
\acro{DOM}{Document Object Model}
\acro{REST}{Representational State Transfer}

\end{acronym}

	\newpage
	
	% Abbildungsverzeichnis
	\cleardoublepage
	\phantomsection \label{listoffig}
	\addcontentsline{toc}{chapter}{Abbildungsverzeichnis}
	\listoffigures
	\newpage	
	
	\renewcommand{\thepage}{\arabic{page}}
	\setcounter{page}{1}
	
	% Hier den Inhalt einfügen
	%\input{kapitel/kapitel01}


	% Literaturverzeichnis
	\cleardoublepage
	\phantomsection \label{listoflit}
	\addcontentsline{toc}{chapter}{Literaturverzeichnis}
	\begin{thebibliography}{\hspace{25mm}}


\bibitem[Backbone]{bip:BackboneEntwickeln}
	\textsc{Osmani, Addy (2013: Developing Backbone.js Applications}\\
	\textbf{First Revision, O'Reilly, Sebastopol}\\
	Einsichtnahme: 11.08.2014

	
\bibitem[Businessweek Geschichte Smartphones]{bip:BW}
	\url{http://www.businessweek.com/articles/2012-06-29/before-iphone-and-android-came-simon-the-		firstsmartphone}\\
	Einsichtnahme	25.08.2014
	
	
\end{thebibliography}

	\newpage
	
	% Erklärung
	\input{ads/erklaerung}
	\newpage

\end{document}
